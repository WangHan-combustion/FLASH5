\documentclass{article}
\usepackage{graphicx,color}
\usepackage{framed}
\usepackage{amsmath}
\setlength{\FrameSep}{2pt}
\usepackage[tight,footnotesize]{subfigure}
\usepackage{color}

\usepackage{url}
\usepackage{algorithmic}
\usepackage[linesnumbered,ruled,commentsnumbered]{algorithm2e}

\newcommand{\code}[1]{{\tt#1}}

\newcommand{\mynote}[1]{\textcolor{blue}{{\bf#1}}}
\newcommand{\reply}[1]{\textcolor{green}{{\bf#1}}}

\newcommand{\flash}{{\it FLASH}\xspace}
\newcommand{\amrex}{{\it AMReX}\xspace}
\newcommand{\paramesh}{{\it Paramesh}\xspace}
\newcommand{\castro}{{\it Castro}\xspace}
\newcommand{\clash}{{\it Clash}\xspace}
\newcommand{\chimera}{{\it Chimera}\xspace}
\newcommand{\grid}{{\it Grid}\xspace}
\newcommand{\driver}{{\it Driver}\xspace}

\title{CLASH API and Code Changes Required in
Component Codes}

\begin{document}
\maketitle
\section{Introduction}
\label{sec:introduction}
This report documents modifications to the suite of codes that form
the eco-system of simulation capabilities for running supernovae
models. From the software perspective, the objective of the project is
to bring together a group of codes that have existed in varying
degrees of isolation to a state where they can utilize a union of
capabilities available in all of them. The key aspect of this exercise
is to convert individual capabilities into a set of composable
components where all the commonly needed capabilities interoperate
with one another. \amrex provides the common adaptive mesh refinement
infrastructure that handles both the discretization and the runtime
support for all configurations of the combined code eco-system. 
\chimera brings microphysics capablities to the eco-system, while
\flash and \castro provide two different ways of composing the
capabilities into a whole application. \flash and \castro have some
duplicated capabilities and some exclusive ones. Since they both have
their own user bases already, we maintain backward compatibility with
their own modes of configuring from the available capabilities. We
achieve this by defining APIs and documenting data crossing
boundaries along with any transformations needed for data format
compatibility and implementing wrapper layers that do the needed
transformations. By documenting the needed modifications and the APIs
this report meets the requirements of the Q3 milestone of the ECP
Astro project. 

\section{AMRex}
\label{sec:amrex}
\amrex  will be the common underlying adaptive mesh management
infrastructure for \clash. \amrex is derived from BoxLib \cite{boxlib}
with a view to masking the impact of hyperparallelism and
asynchronicity from the applications codes. Thus \amrex will
provide high level interfaces for querying the mesh and interacting
with it while shielding the client code from the details of managing
the mesh for different computing platforms. Additionally, with the
same underlying infrastructure \amrex will present different flavors
of AMR formulations to the code. For example, \castro assumes that it
is working with a patch-based meshing package with refinement in time.
On the other hand \flash assumes octree based AMR, and retaining that
functionality is important for many of its solvers. Therefore, \amrex
will implement Fortran iterators that mimic both of these
behaviors. \amrex will also provide interfaces with other solver
libraries such as PETSc and hypre; the other codes will modify their
access to these libraries accordingly. 
\subsection{Interfaces}
\label{sec:amrex-interfaces}
The critical interfaces needed to obtain necessary services from the
AMR packages are
\begin{itemize}
\item \code{amrex\_init\_from\_scratch} -- build the AMR mesh from
  scratch
\item \code{ amrex\_amrcore\_init} -- setup the core amr functionality
\item \code{call amrex\_amrcore\_finalize} -- complete everything and
  release all amrcore resources
\item \code{amrex\_octree\_init} -- initialize \amrex in octree mode
\item \code{amrex\_octree\_finalize} -- finalize the octree mode
\item \code{amrex\_regrid} -- method to regrid the mesh
\item \code{amrex\_multifab\_build} -- build a union of boxes at a given level
of refinement
\item \code{amrex\_multifab\_swap} -- allows evolution to have
  not-in-place update without having to copy data, alternative
  timesteps switch their source and destination data structures
\item \code{amrex\_multifab\_destroy} -- release resources from multifab
  and destroy
\item \code{amrex\_fluxregister\_build} -- make provision for flux
  registers at fine-coarse boundaries
\item \code{amrex\_fluxregister\_destroy} -- release flux 
  registers at fine-coarse boundaries
\item \code{amrex\_mfiter\_build} -- build the iterator 
\item \code{amrex\_octree\_iter\_build} -- build the iterator with octree like behaviour
\item \code{amrex\_mfiter\_destroy} -- destroy the iterator
\item \code{amrex\_data\_init} -- allocate space for state data
\item \code{amrex\_data\_finalize} -- release state data space
\item \code{amrex\_fillpatch} -- fill ghost cells with appropriate values
  from neighbors at the same level or from the underlying coarser
  level
\item \code{amrex\_fillcoarsepatch} -- fill a coarse patch from fine values
\item \code{flux\_register} -- update flux registers where needed
\end{itemize}


\section{Castro}
\label{sec:castro}
\castro needs relatively modest changes because it was built on top
of BoxLib. It will not need fundamental
changes to how it interacts with the mesh and state variables. Main
changes needed in \castro are listed below.
\begin{itemize}
\item Wrappers for calling \chimera  interfaces.
\item For using any capability from \flash,  wrappers are needed for
calling the initialization functions for the corresponding unit,
which can also transmit all the parameters and static data needed for
the solver operation. These functions initialize all runtime parameters, 
scratch space and other unit scope scalar variables.
\end{itemize}

\section{Chimera}
All of \chimera's code modules will be transformed into encapsulated
components that are usable by both \castro and \flash. All the modules
have pointwise computations which obviates the need to know any part of
the physical mesh for them. They need runtime initializations and
various interaction tables are tied to specific types of equations of
state. Information about both runtime parameters and the dependent EOS
implementation will be encoded in a meta-information file (Config for
\flash) and ... for \castro.
\subsection{Interfaces}
\label{sec:chimera-interfaces}
\begin{itemize}
\item
\end{itemize}

\section{FLASH}
\label{sec:flash}
 In \flash some changes are trivial but pervasive, some others are
modest and fairly extensive, and some are deep but not very
extensive. The deepest change is enabling \amrex to manage mesh
refinement and state variables instead of \paramesh. Even though
\amrex will mimic the octree like behavior of \paramesh, there are
fundamental differences in the way that state variables and metadata
are managed and made available to other code units. Below we list the
changes, the depth and extent of the changes, and the reasoning behind
the design choice.

\subsection{AMR Specific Modifications}
\label{sec:amr}
\begin{enumerate}
\item Convert looping over blocks to {\it do while} loops using
iterators -- change is trivial but pervasive.
\begin{itemize}
\item Current method fetches list of blocks and their count from the Grid Unit.
\item This method is less flexible for scheduling individual blocks
and precludes the possibility of working on sub-blocks as tiles.
\item Iterators are more effective at hiding details of tiling or
asynchronous runtime task management. 
\end{itemize}
\item Eliminate blockID as the method of identifying blocks and for
fetching their meta information -- change is modest but pervasive.
\begin{itemize}
\item \paramesh knows its blocks through blockID, but \amrex does not.
\item Currently all physics units use the blockID to fetch information
such as physical coordinates, dx's etc from the \grid unit in a pull
mode. 
\item In the new mode this information comes as attached metadata with
the iterator, which further abstracts mesh management details from physics.
\end{itemize}
\item Move looping over blocks from individual physics units to the
\driver unit where timestepping is implemented -- change is trivial
but pervasive.
\begin{itemize}
\item Any future changes to the method of looping over blocks will be localized instead
of being pervasive.
\item It lets us keep the possibility of time refinement open without
having to make another pervasive change to the code.
\item Currently the code employs operator splitting. If some model
requires fusing two operators it would be impossible due to
encapsulation in the current method which includes looping over blocks.
\item Letting the iterators work in the \driver allows operators fusion
possibility while retaining the physics encapsulation intact because
the encapsulation pertains to a single block/tile invocation of the physics.
\end{itemize}
\item Change the way that the intial mesh is generated and the initial
conditions are applied -- change is local and modest.
\begin{itemize}
\item Small changes are needed in the current method for continuing
to use \paramesh.
\item A new implementation is needed for initializing with \amrex.
\end{itemize}
\item Create an \amrex equivalent iterator for \paramesh for backward
compatibility -- change is local and modest.
\item Several helper and accessor functions need new implementation within the
\grid unit for interfacing with \amrex -- change is local but
relatively deep in the form of an alternative implementation of the
\grid unit functionality.
\item Handling of flux conservation will change -- the change is local
and modest.
\end{enumerate}

\subsection{Other Modifications}
\label{sec:other}
\begin{enumerate}
\item Implement {\it Unit\_init} methods for all new capabilities or
alternative implementations being used from \castro and \chimera along
with the necessary wrappers -- change is local and will vary in
complexity from unit to unit.
\item Implement functions to create arbitrarily sized vectors for
operating on -- function implementation is modest effort, but being
able to use it is likely to require deep changes in many parts of the
code.
\begin{itemize}
\item This ability is needed for utilizing hardware that requires
vectorization, e.g. accelerators.
\item Initially we will use it for zero-dimensional micro-physics, but
eventually this may need to become embedded in the \amrex
infrastructure.
\item One other major change is in I/O. Currently I/O in \flash gets
direct access to all \grid data. This will now have to be obtained
through the iterator.
\end{itemize}
\item \flash's new version will allow the iterator to loop over blocks
per level. This is the default mode for \amrex, and it leaves open the
possibility of refinement in time in \flash.
\end{enumerate}

\subsection{Interfaces}
\label{sec:flash-interfaces}
In \flash the knowledge of {\it blockid} is pervasive across the
entire code. \amrex, on the other hand, has no such notion. Thus many
of existing interfaces have to be modified where information that came
form querying the mesh for metadata about a block thought its identity
is now either passed in, or acquired through some other means. Also,
because the metadata transfer changes from a pull mode, where the mesh
is queried for information, to push mode, where mesh hands over
information with the iterator, many interfaces become redundant. Below
we list new interfaces added to \flash and also the modified and
eliminated interfaces. The list is not exhaustive, it ignores smaller
helper functions, and may grow as more of combined capabilities
undergo verification. We also do not list all local interfaces within
a FLASH unit that need to change because they are too many, and are
never exposed outside the unit.

\subsubsection*{New interfaces}
\begin{itemize}
\item \code{Grid\_makeVector} -- turn either a subset of a block into
  a flat vector or coalesce multiple blocks into a single vector, and
  also the reverse
\item \code{Grid\_multivab\_build} --mimic \amrex multifab behavior
  over \paramesh
\item \code{Grid\_mviter\_build} -- build iterator on top of the above
  mimiced behavior in \paramesh 
\end{itemize}
\subsubsection*{Modified interfaces}
\begin{itemize}
\item \code{Grid\_getCellCoords} -- fetch physical coordinates 
\item \code{Grid\_getDelta} -- fetch dx,dy,dz values
\item \code{Unit} -- For all physics units, their main driver
  interface, for example Hydro, Cool etc.
\item \code{Unit\_computeDt} -- For all physics units that contribute
  to determination of dt their compute dt interface
\item \code{Simulation\_initBlock} -- apply initial conditions to one
  block of data
\item \code{Eos\_wrapped} -- apply equation of state to collection of data.
\item \code{Eos\_guardCells} -- apply equation of state to ghost cells
  filled at fine-coarse levels
\end{itemize}
\subsubsection*{Eliminated interfaces}
\begin{itemize}
\item \code{Grid\_getLocalNumBlks}
\item \code{Grid\_getListofBlocks}
\item \code{Grid\_getBlkIndexLimits}
\item \code{Grid\_getBlkPtr}
\item \code{Grid\_releaseBlkPtr}
\item \code{Grid\_[get/put][point/row/plane/Blk]Data}
\item \code{Grid\_BlkRefineLevel}
\item \code{Grid\_getBlkCornerId}
\item \code{Grid\_getBlkBoundBox}
\end{itemize}

\end{document}
