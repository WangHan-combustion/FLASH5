\documentclass{article}
\usepackage{amsmath}
\usepackage{fancyvrb,relsize}
\usepackage{graphicx}
\usepackage[bookmarks=true,bookmarksnumbered=true]{hyperref}

\setlength\topmargin{0in}
\setlength\headheight{0in}
\setlength\headsep{0in}
\setlength\textheight{7.7in}
\setlength\textwidth{6.5in}
\setlength\oddsidemargin{0in}
\setlength\evensidemargin{0in}
\setlength\parindent{0.25in}
\setlength\parskip{0.25in} 



\begin{document}

\title{FLASH I/O}
%\author{Chris Daley}
\maketitle

\tableofcontents
%\listoftables
%\listoffigures


\newpage
\section{Data stored in memory}
FLASH stores double precision cell-centered mesh data in a
5-dimensional array named \textbf{unk}.  In Fortran ordering the array
has size:

\begin{align} \label{Eq:unk}
     unk ( &NUNK\_VARS, \nonumber \\
     & NXB+2*NGUARD, \nonumber \\
     & NYB+2*NGUARD, \nonumber \\
     & NZB+2*NGUARD, \nonumber \\
     & MAXBLOCKS)
\end{align}

\textbf{NUNK\_VARS} is the number of cell-centered variables, where a
variable (in FLASH terminology) is e.g. density (dens), pressure
(pres), temperature(temp).  \textbf{NXB}, \textbf{NYB}, \textbf{NZB}
are the number of x,y,z internal cells, and \textbf{NGUARD} is the
number of guard cells.  Finally, \textbf{MAXBLOCKS} is the maximum
number of blocks that can reside in a single MPI process.

This array is allocated once at the start of a FLASH application and
has the same size in each MPI process.  Since FLASH decomposes the
global domain across blocks, the maximum number of blocks that can
exist is \textbf{MAXBLOCKS} * \textbf{NPROCS}, where \textbf{NPROCS}
is the number of MPI processes.


\section{File types}
FLASH generates checkpoint file and plot files.  A checkpoint file
contains all variables in double precision and a plot file contains a
subset of variables in single precision.


\section{File formats}
\subsection{Standard file format}
\label{sec:standardFile}
FLASH writes each variable into a separate dataset (in HDF5
terminology).  We only store data for blocks in existance and exclude
guard cells.  For example, an application that uses two variables
\textbf{dens} and \textbf{pres} writes to \textbf{dens} and
\textbf{pres} datasets:

\begin{align} \label{Eq:file1dens}
     dens (& 1:NXB, \nonumber \\
     & 1:NYB, \nonumber \\
     & 1:NZB, \nonumber \\
     & 1:nblocks\_global)
\end{align}

\begin{align} \label{Eq:file1pres}
     pres (& 1:NXB, \nonumber \\
     & 1:NYB, \nonumber \\
     & 1:NZB, \nonumber \\
     & 1:nblocks\_global)
\end{align}

Here, \textbf{nblocks\_global} is the global number of blocks in
existance.  


\subsection{Experimental file format}
\label{sec:experimentalFile}
FLASH writes all variables into the same dataset.  Again, we only
store data for blocks in existance and exclude guard cells.  An
application using two variables writes the following data to the
\textbf{unk\_file} dataset:

\begin{align} \label{Eq:file2}
     unk\_file (& 1:2, \nonumber \\
     & 1:NXB, \nonumber \\
     & 1:NYB, \nonumber \\
     & 1:NZB, \nonumber \\
     & 1:nblocks\_global)
\end{align}


\section{I/O strategies}
\subsection{Standard blocking I/O}
\label{sec:standardBlocking}
The standard I/O implementation (present in FLASH2 and FLASH3) stores
data in the file format shown in Section \ref{sec:standardFile}.  For
a checkpoint file, FLASH copies a single variable from \textbf{unk}
into a temporary double precision array named \textbf{unkBuf}.  The
data in \textbf{unkBuf} is written to file and then the next variable
is stored in \textbf{unkBuf}, and the process repeats.  Note that
\textbf{unkBuf} is only large enough to store the internal cells for a
single variable at a time.

\begin{align} \label{Eq:strat1chk}
     unkBuf ( &1:NXB, \nonumber \\
     & 1:NYB, \nonumber \\
     & 1:NZB, \nonumber \\
     & 1:nblocks) = \nonumber \\
     unk (&i, \nonumber \\
     & NGUARD+1:NXB+NGUARD, \nonumber \\
     & NGUARD+1:NYB+NGUARD, \nonumber \\
     & NGUARD+1:NZB+NGUARD, \nonumber \\
     & 1:nblocks)
\end{align}

Here, \textbf{nblocks} is the number of blocks in existance in each
MPI process.  This value is generally different in different MPI
processes.  \textbf{i} is a loop index that iterates $1$ to
\textbf{NUNK\_VARS}.\newline

A second temporary single precision array named \textbf{unkt} is used
for plot file data.

\begin{align} \label{Eq:strat1plot}
     unkt ( &1:NXB, \nonumber \\
     & 1:NYB, \nonumber \\
     & 1:NZB, \nonumber \\
     & 1:nblocks) = \nonumber \\
     real (unk (&j, \nonumber \\
     & NGUARD+1:NXB+NGUARD, \nonumber \\
     & NGUARD+1:NYB+NGUARD, \nonumber \\
     & NGUARD+1:NZB+NGUARD, \nonumber \\
     & 1:nblocks) kind = single)
\end{align}

Here, \textbf{j} is a placeholder for each variable that is selected
to be written to plot file.


\subsection{Experimental blocking I/O}
\label{sec:experimentalBlocking}
The experimental I/O implementation stores data in the file format
shown in Section \ref{sec:experimentalFile}.  It uses MPI derived
datatypes to describe a single FLASH block, which is then used to
pick-up the required data in \textbf{unk} instead of using temporary
arrays.  Note that all blocks have the same number of variables and
the same number of cells in the x,y,z direction

The block datatype used for checkpoint files excludes guardcells:

\begin{align} \label{Eq:strat2chk}
     block\_chk (& 1:NUNK\_VARS, \nonumber \\
     & NGUARD+1:NXB+NGUARD, \nonumber \\
     & NGUARD+1:NYB+NGUARD, \nonumber \\
     & NGUARD+1:NZB+NGUARD)
\end{align}

Each MPI process selects the data to be written to file by using a
count of \textbf{nblocks} instances of MPI datatype
\textbf{block\_chk} at starting point \textbf{unk(1,1,1,1,1)}.

The block datatype used for plot files is more complicated, as it
excludes guardcells and only includes variables of interest:

\begin{align} \label{Eq:strat2plot}
     block\_plot (& subset, \nonumber \\
     & NGUARD+1:NXB+NGUARD, \nonumber \\
     & NGUARD+1:NYB+NGUARD, \nonumber \\
     & NGUARD+1:NZB+NGUARD)
\end{align}

Here, \textbf{subset} indicates the chosen subset of variables from
space \textbf{1:NUNK\_VARS} that may or may not be contiguous.  Once
again, each process selects the data to be written to file by using a
count of \textbf{nblocks} instances of MPI datatype
\textbf{block\_plot} at starting point \textbf{unk(1,1,1,1,1)}.


\subsection{Standard non-blocking I/O}
\label{sec:standardNonBlocking}
This implementation produces the same files as the implementation
described in Section \ref{sec:standardBlocking}.  It uses a MPI
derived datatype to select a single variable at a time.

\begin{align} \label{Eq:strat3}
     block\_var (&1, \nonumber \\
     & NGUARD+1:NXB+NGUARD, \nonumber \\
     & NGUARD+1:NYB+NGUARD, \nonumber \\
     & NGUARD+1:NZB+NGUARD)
\end{align}

The first array index in the MPI derived datatype \textbf{block\_var}
picks up a single variable from space \textbf{1:NUNK\_VARS}.  The same
datatype is used to select data in memory for both checkpoint and plot
files. \newline

Suppose a FLASH application includes two variables \textbf{dens} and
\textbf{pres}.  Each process writes to the \textbf{dens} dataset by
using a count of \textbf{nblocks} instances of MPI datatype
\textbf{block\_var} at starting point \textbf{unk(1,1,1,1,1)}.  Then
each process writes to the \textbf{pres} dataset by using a count of
\textbf{nblocks} instances of MPI datatype \textbf{block\_var} at
starting point \textbf{unk(2,1,1,1,1)}.
\end{document}
